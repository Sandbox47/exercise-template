
% Überschreibt enumerate Befehl, sodass 1. Ebene Items mit
\renewcommand{\theenumi}{(\alph{enumi})}
% (a), (b), etc. nummeriert werden.
\renewcommand{\labelenumi}{\text{\theenumi}}

% \setcounter{countername}{number}: Legt den Wert des Counters fest
% \stepcounter{countername}: Erhöht den Wert des Counters um 1.
\newcounter{sheetnr}
\newcounter{exnum}

\newcommand{\exercise}[1]{\section*{Aufgabe \theexnum\stepcounter{exnum}: #1}}

\newcommand{\mbeq}{\overset{!}{=}}
\newcommand{\probgets}{\overset{\$}{\gets}}

\renewcommand{\autodot}{}
\renewcommand{\thesubsection}{(\alph{subsection})}

\renewcommand{\theenumi}{\textbf{\arabic{enumi}.}}

\newenvironment{proposition}[1][]{\par\noindent\textit{Proposition}\ifx&#1&\else\ (#1)\fi. }

\ohead{
    Bj\"orn Aheimer 3594348 st177191@stud.uni-stuttgart.de \\ 
    Felix R\"ohr 3567731 st176436@stud.uni-stuttgart.de \\
    Aleksis Vezenkov 3607798 st179084@stud.uni-stuttgart.de}
    
% \chead{} kann mittleren Kopfzeilen Teil sezten
% \ihead: Setzt linken Teil der Kopfzeile mit
% Modulnamen, Semester und Übungsblattnummer
\ihead{Module Name\\
WS 2023/24\\
Homework \thesheetnr}